\section{Mathematical Proof and Verification}\label{appendix:math}
\subsection{The OLS Estimator of $\pmb{\hat{\beta}}$}
\label{appendix:ols}

In this appendix, we introduce the basic mathematics to obtain a formula for the \acs{OLS} estimators $\hat{\beta_{i}}$ in \acs{MLR}. We reach the goal through minimizing the \hyperref[appendix:notations]{\text{RSS}}, given $Y_i$ and $\hat{Y_i}$ in Equation (\ref{eq:mlr_srf}). \ie\,We take partial derivatives of \hyperref[appendix:notations]{\text{RSS}} with respect to each $\hat{\beta_{i}}$, set those derivatives equal to zero, and solve for $\hat{\beta_{i}}$s:
\[
\frac{\partial RSS}{\partial \hat{\beta_{0}}} = 0 \quad
\frac{\partial RSS}{\partial \hat{\beta_{1}}} = 0 \quad
\dotsm \quad
\frac{\partial RSS}{\partial \hat{\beta_{k}}} = 0
\]

To show the properties of this system of linear equations, let us check the equation for $\hat{\beta_{0}}$:
\[
\frac{\partial RSS}{\partial \hat{\beta_{0}}}=(-2)\sum \big[\,Y_{i}-(\,\hat{\beta_{0}}+\hat{\beta_{1}}X_{1i}+\dots+\hat{\beta_{n}}X_{ki}\,)\,\big]=0
\]

\ie 
$$n\hat{\beta_{0}}+\big(\sum X_{1i}\big)\hat{\beta_{1}}+\big(\sum X_{2i}\big)\hat{\beta_{2}}+\dots+\big(\sum X_{ki}\big)\hat{\beta_{k}}=\sum Y_{i}$$

Similarly, we get a system of linear equations for other $\hat{\beta_{i}}$s:
\begin{equation}\label{eq:ols_sol}
\begin{aligned}
\big(\sum X_{1i}\big)\hat{\beta_{0}}+\big(\sum X_{1i}^2\big)\hat{\beta_{1}}+\big(\sum X_{1i}X_{2i}\big)\hat{\beta_{2}}+\dots+\big(\sum X_{1i}X_{ki}\big)\hat{\beta_{k}}&=\sum X_{1i}Y_{i}
\\
\big(\sum X_{2i}\big)\hat{\beta_{0}}+\big(\sum X_{1i}X_{2i}\big)\hat{\beta_{1}}+\big(\sum X_{2i}^2\big)\hat{\beta_{2}}+\dots+\big(\sum X_{2i}X_{ki}\big)\hat{\beta_{k}}&=\sum X_{2i}Y_{i}
\\
\vdotswithin{=}
\\
\big(\sum X_{ki}\big)\hat{\beta_{0}}+\big(\sum X_{1i}X_{ki}\big)\hat{\beta_{1}}+\big(\sum X_{2i}X_{ki}\big)\hat{\beta_{2}}+\dots+\big(\sum X_{ki}^2\big)\hat{\beta_{k}}&=\sum X_{ki}Y_{i}
\end{aligned} 
\end{equation}


A simple way to solve Equations (\ref{eq:ols_sol}) is by expressing them in matrix form. Before moving forward, it is essential to know the matrix representation of our \acs{PRF} in Equation (\ref{eq:mlr_prf}):
\begin{equation}\label{eq:ols_mat}
\pmb{Y} = \pmb{X}\pmb{\beta}+\pmb{\epsilon}\eqcomma
\end{equation}
where
\[
\pmb{Y}=\begin{pmatrix}
y_{1}\\y_{2}\\y_{3}\\\vdots\\y_{n}
\end{pmatrix}
\quad
\pmb{X}=\begin{pmatrix}
1 & x_{11} & x_{21} & \dots & x_{k1}\\
1 & x_{12} & x_{22} & \dots & x_{k2}\\
1 & x_{13} & x_{23} & \dots & x_{k3}\\
\vdots & \vdots & \vdots & \ddots & \vdots\\
1 & x_{1n} & x_{2n} & \dots & x_{kn}\\
\end{pmatrix}
\quad
\pmb{\beta}=\begin{pmatrix}
\beta_{0}\\\beta_{1}\\\beta_{2}\\\vdots\\\beta_{k}
\end{pmatrix}
\quad
\pmb{\epsilon}=\begin{pmatrix}
\epsilon_{1}\\\epsilon_{2}\\\epsilon_{3}\\\vdots\\\epsilon_{n}
\end{pmatrix}
\]


Thus, by similar transformation, the system of linear Equations (\ref{eq:ols_sol}) can be written as:
\begin{equation}\label{eq:ols_sol2}
(\pmb{X}^T\pmb{X})\pmb{\hat{\beta}}=\pmb{X}^T\pmb{Y}
\end{equation}

Thus, the solutions of ${\hat{\beta_{i}}}$ in Equation (\ref{eq:ols_sol2}) can be obtained if $\pmb{X}^T\pmb{X}$ is invertible\footnote{In linear algebra, an $n$-by-$n$ matrix $\pmb{A}$ is called invertible if there exists an $n$-by-$n$ matrix $\pmb{B}$ such that $\pmb{AB}=\pmb{BA}=\pmb{I_{n}}$, where $\pmb{I_{n}}$ is $n$-by-$n$ identity matrix. We say that $\pmb{B}$ is the inverse of $\pmb{A}$, denoted by $\pmb{A^{-1}}$.}:
\begin{equation}\label{eq:ols_sol3}
\pmb{\hat{\beta}}=(\pmb{X}^T\pmb{X})^{-1}\pmb{X}^T\pmb{Y}
\end{equation}

\section{Conclusion}
\label{sec:conclusion}

In this paper, we present a regression analysis using classical econometrics methods. Namely, we use \acs{MLR} and \acs{MNR} models as our functional form based on economic theories, literature and intuition. We estimate the parameters using the \acs{OLS} estimation method. Though not all \acs{OLS} assumptions are satisfied, as shown in Appendix \ref{appendix: assumptions}, we believe the drawbacks of the regression models will not interfere with our interpretations significantly because \acs{OLS} estimators are not \acs{BLUE} in a lot of economics applications. As long as the equations are supported by sound theory, free of major econometric problems, used theoretically logical functional form and no obvious important variables omitted, we can evaluate the regression results as useful \cite{studenmund_2017}. This proves our model accuracy when we interpret the results in the best-fitted model Regression (5). 

In econometrics, subfields of economics are involved such as macro, labor, social economics and government finance \cite{wooldridge_2020}. In this paper, we focus on only limited subjects in economics, but more interesting research can be done in other fields too. Another direction of future investigation will be to consider an empirical analysis of time series data. Especially, we are interested in real GDP and inflation in an economy over a given time period. Since the methods in this paper cannot be used in time series regression, expanding our knowledge to analyze time series data appears to be a natural next step in the near future.
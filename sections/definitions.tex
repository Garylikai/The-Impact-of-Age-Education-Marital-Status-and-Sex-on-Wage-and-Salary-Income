%%% section divider;
%\newcommand{\secdiv}{\begin{center}
%\adforn{21}\quad\adforn{11}\quad\adforn{49}
%\end{center}
%}



%%%% Important sets
\newcommand{\setOfReals}{\mathbb{R}}
\newcommand{\setOfNaturals}{\mathbb{N}}
\newcommand{\setOfNonnegativeIntegers}{{\mathbb{N}_0}}
\newcommand{\setOfPositiveReals}{{\setOfReals_{+}}}
\newcommand{\setN}[1]{ [ #1]}
\newcommand{\setTime}[1]{\mathcal{T}_{#1}}

\newcommand{\borel}[1]{\mathcal{B} (#1 )}
\newcommand{\setOfMeasures}[2]{\mathcal{M}_{#2}\left(#1\right)}
\newcommand{\setOfProbabilityMeasures}[1]{\mathcal{P}\left(#1\right)}
\newcommand{\setOfContinuous}[2]{{C}\left(#1, #2\right)}
\newcommand{\setOfBoundedContinuous}[2]{{C}_{b}\left(#1, #2\right)}
\newcommand{\setOfBoundedDifferentiable}[2]{{C}_{b}^{(1)}\left(#1, #2\right)}
\newcommand{\setOfcadlag}[2]{{D}\left(#1, #2\right)}
\newcommand{\setOfcompactSupported}[2]{C_{K}\left(#1, #2\right)}
\newcommand{\setOfBoundedMeasurable}[2]{{B}_{b}\left(#1, #2\right)}


\newcommand{\BigO}[1]{\mathop{}\!O{\left(#1\right)}}
\newcommand{\smallO}[1]{\mathop{}\!o{\left(#1\right)}}




% generic system names
\newcommand{\sys}[1]{\textsc{#1}}





%\newcommand{\supp}{{\mathrm{supp}}  }
%\def\bD{\boldsymbol\Delta}
%\def\L{{\cal L}}

% \def\C{{\mathbb C}}
% \def\eps{\epsilon}
%\def\S{\ensuremath{\mathrm{S}}}
%\def\I{\ensuremath{\mathrm{I}}}
%\def\R{\ensuremath{\mathrm{R}}}
% \def\var{\mathbb{V}\mathrm{ar}}
% \def\cov{\mathbb{C}\mathrm{ov}}
% \def\corr{\mathbb{C}\mathrm{orr}}
% \def\la{\lambda}
% \def\ga{\gamma}
% \def\Rn{{\cal R}_0}
% \def\io{\iota}
% \def\ro{\Rn}
% \def\tS{\tilde S}
% \def\nn{\nonumber}
% \def\th{\theta}
% \newcommand\ind[1]{\mathbb{I}_{#1}}
%
% \newcommand\be{\begin{equation}\label}
%\newcommand\ee{\end{equation}}
%
% \def\G{{\cal G}}
% \def\d{\partial}
%
% \def\c{\circ}
% % % % Probability distributions

% % % Distributions
\newcommand{\Bin}[2]{\sys{Binomial}(#1, #2 )}
\newcommand{\Poi}[1]{\sys{Poisson} (#1)}


% % % % Functions
\newcommand{\measureIntegral}[2]{#1\left(#2 \right) }
%\newcommand{\measureIntegral}[2]{\langle#1, #2 \rangle }

\newcommand{\optionalVariation}[1]{ [#1] }
\newcommand{\predictableVariation}[1]{ \langle#1\rangle }

\newcommand{\cardinality}[1]{\mid #1  \mid}

\DeclarePairedDelimiter\ceil{\lceil}{\rceil}
\DeclarePairedDelimiter\floor{\lfloor}{\rfloor}

\DeclarePairedDelimiter\abs{\lvert}{\rvert}
% Swap the definition of \abs* and \norm*, so that \abs
% and \norm resizes the size of the brackets, and the
% starred version does not.
\makeatletter
\let\oldabs\abs
\def\abs{\@ifstar{\oldabs}{\oldabs*}}
%

\newcommand{\cadlag}{c\`adl\`ag\,}

\newcommand{\aggregation}[1]{\mathsf{agg}\left( #1  \right)}
\newcommand{\uniformization}[1]{\mathsf{unif}\left( #1  \right)}
\newcommand{\fibre}[1]{\mathsf{fibre}\left( #1 \right)}

\newcommand{\SymmetryGroup}[1]{\mathsf{Sym}\left( #1 \right)}

\newcommand{\Aut}[1]{\mathsf{Aut}\left( #1 \right)}

\newcommand{\coset}[2]{\langle #1 \rangle_{#2}}

\newcommand{\KL}[2]{D_\mathrm{KL}\left( #1 \mid\mid #2  \right) }


%\newcommand{\metricGivenSpace}[3]{\rho_{#3} \left( #1, #2  \right) }
\newcommand{\metricGivenSpace}[1]{\mathsf{d}_{#1} }

\newcommand{\etal}{\textit{et al.}}

\newcommand{\wrt}{\textit{w.r.t.\ }}


\newcommand{\clip}[1]{\,\mathsf{clip} \left(#1 \right)}

\renewcommand{\vec}[1]{\mathbf{#1}}
\newcommand{\mat}[1]{\mathrm{\mathbf{#1}}}
\newcommand{\Cov}[1]{\mathsf{Cov}(#1)}
\newcommand{\Corr}[1]{\mathsf{Corr}(#1)}
\newcommand{\diag}[1]{\mathrm{diag}(#1)}

\DeclareMathOperator*{\esssup}{ess\,sup}


\newcommand{\defeq}{\coloneqq}

\newcommand{\identity}{\mathsf{Id}}

\newcommand{\indicator}[1]{\mathsf{1}_{#1}}

\newcommand{\FOperator}[2]{\mathsf{A}_{#2} (#1) }
\newcommand{\HOperator}[2]{\mathsf{C}_{#2} ( #1) }
\newcommand{\GOperator}[2]{\mathsf{B}_{#2} ( #1) }


\newcommand{\JOperator}[1]{\mathsf{J}( #1) }
\newcommand{\UOperator}[1]{\mathsf{U}( #1) }
\newcommand{\VOperator}[1]{\mathsf{V}( #1) }

\newcommand{\myOperator}[1]{\mathsf{#1}}


\newcommand{\Permanent}[1]{\mathsf{per} \hspace{2pt} #1}
\newcommand{\allpermutations}[1]{\Theta (#1)}
\newcommand{\diophantine}[2]{\Lambda (#1,#2)}

\newcommand{\projection}[2]{\mathbbm{P}_{#1}{\left( #2 \right)}}

\newcommand{\diffOperator}[2]{\mathbbm{D}^{#2}#1}
\newcommand{\differential}[1]{\,\mathrm{d} #1}
\newcommand{\timeDerivative}[1]{\frac{\differential}{\differential t} #1 }
\newcommand{\partialDerivative}[2]{\frac{\partial}{\partial #2} #1 }

\newcommand{\interior}[1]{ \mathsf{Int} \, #1 }
%\newcommand{\closure}[1]{ \mathsf{Cl} \, #1 }
\newcommand{\closure}[1]{ \overline{ #1 } }
%\newcommand{\boundary}[1]{\mathsf{Bnd}\, #1}
\newcommand{\effectiveDomain}[1]{\mathsf{dom}\, #1}

\newcommand{\range}[1]{ \mathsf{rng} \left(#1\right)  }

\newcommand{\eqstop}{.}
\newcommand{\eqcomma}{,}


\newcommand{\norm}[2]{\left\lVert#1\right\rVert_{#2}}

\newcommand{\E}{\mathsf{E}}
\newcommand{\Eof}[1]{\E\left[#1 \right]}
\newcommand{\EofP}[2]{\E_{#1}\left(#2\right)}

\newcommand{\V}{\mathsf{Var}}
\newcommand{\VarOf}[1]{\V\left[#1\right]}
\newcommand{\VarOfP}[2]{\V_{#1}\left(#2\right)}

\newcommand{\NPermuteR}[2]{ (  #1)_{#2}  }

\newcommand{\prob}{\mathsf{P}}
\newcommand{\probOf}[1]{\prob\left(#1\right)}

\newcommand{\given}{\mid}

\newcommand{\history}[1]{\mathcal{F}_{#1}  }

\newcommand{\trans}{^{\mathsf T}}

\newcommand{\disteq}{\, \overset{\mathcal{D}}= \, }

\newcommand{\ConvInProb}{\xrightarrow[]{ \hspace*{4pt}  \text{         P   } }}
\newcommand{\ConvAlmostSure}{\xrightarrow[]{ \hspace*{4pt}  \text{         a.s.   } }}
\newcommand{\ConvInDist}{  \overset{\hspace*{4pt}  \mathcal{D}  }{\implies } }


\DeclareMathOperator*{\Bigcdot}{\bullet}


\newcommand{\myExp}[1]{\exp \left( #1 \right)  }

\newcommand{\ie}{\textit{i.e.}}
\newcommand{\eg}{\textit{e.g.}}
\newcommand{\ia}{\textit{i.a.}}
\newcommand{\viz}{\textit{viz.}}
\newcommand{\vide}{\textit{vide}}

\newcommand{\quotes}[1]{``#1''}




% TODO-Package with author-macroscd
% \usepackage{todonotes}
% \usepackage{ifthen}
%  \newboolean{draftversion}
% \setboolean{draftversion}{true} %set false for removing comments and watermark
%
%
 %\newcommand{\wasiur}[2]{\ifthenelse{\boolean{draftversion}}{\todo[inline, color=tud1a!50, caption={2do}, #1]{\begin{minipage}{\textwidth-4pt}\emph{Remark Wasiur:}\\#2\end{minipage}}}{}}
%
% \newcommand{\eben}[1]{\ifthenelse{\boolean{draftversion}}{\todo[inline, color=tud1a!50, caption={2do}, #1]{\begin{minipage}{\textwidth-4pt}\emph{Remark Eben:}\\#2\end{minipage}}}{}}
%
% \newcommand{\greg}[1]{\ifthenelse{\boolean{draftversion}}{\todo[inline, color=tud1a!50, caption={2do}, #1]{\begin{minipage}{\textwidth-4pt}\emph{Remark Greg:}\\#2\end{minipage}}}{}}
%
%
%
% \ifthenelse{\boolean{draftversion}}{\usepackage[firstpage]{draftwatermark}}{}

\newcommand{\eben}[1]{\textcolor{tud7a}{$\langle${\slshape{\bfseries Eben:} #1}$\rangle$}}
\newcommand{\wasiur}[1]{\textcolor{tud9d}{$\langle${\slshape{\bfseries Wasiur:} #1}$\rangle$}}
\newcommand{\gr}[1]{\textcolor{tud3a}{$\langle${\slshape{\bfseries Greg:} #1}$\rangle$}}



%%% equation numering
\renewcommand {\theequation}{\arabic{section}.\arabic{equation}}

\pagestyle{plain}
\pagenumbering{arabic}

% \titlespacing\section{0pt}{10pt plus 4pt minus 2pt}{6pt plus 2pt minus 2pt}
% \titlespacing\subsection{0pt}{8pt plus 4pt minus 2pt}{3pt plus 2pt minus 2pt}
% \titlespacing\subsubsection{0pt}{6pt plus 4pt minus 2pt}{2pt plus 2pt minus 2pt}
% \titlespacing{\paragraph}{0pt}{5pt plus 2pt minus 2pt}{2pt plus 2pt minus 2pt}
% %\setlength\parindent{.1 cm}


\newcounter{ocounter}
\newcounter{acounter}
\setlength{\parskip}{0pt}

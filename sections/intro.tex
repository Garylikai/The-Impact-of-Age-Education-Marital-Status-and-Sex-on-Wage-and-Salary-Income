\section{Introduction}
\label{sec:intro}

Nowadays, people may wonder about factors that can lead to higher salaries. Economists have already done lots of empirical analysis on this topic. The objective of the paper is to perform similar empirical research among multiple interested predictor variables \textit{Age}\footnote{We \textit{italicize} the variables in our model in the paper.}, \textit{Education}, \textit{Male}, \textit{Nonmarried}, \textit{US}, and a predicted variable \textit{Wage} using a data set from \acs{CPS}. We treat ordinal variables approximately continuous in our regression model. By performing such a regression analysis, we can analyze and interpret how selected factors can affect salary. More importantly, we, or economists, are able to explain social, economic issues such as gender, birthplace discrimination, income distribution, and the importance of education to future earnings. Thus, this empirical research provides meaningful insights on applications to economics and statistics.

For the rest of the paper, we first filter out irrelevant variables and observations from the data in \hyperref[sec:data]{\text{Data}} Section. Then, we shall provide a rough estimate of what the fittest model looks like in \hyperref[sec:em]{\text{Empirical Methodology}} Section.  Starting from assuming an \acs{MLR} model, we expand our regression model to \acs{MNR} and measure its goodness of fit. Then, we analyze the data and select a model that represents our data set most effectively and accurately using \acs{OLS} estimators obatined from Stata, shown in \hyperref[sec:results]{\text{Results}} Section. Also, we conclude the statistical significance of the coefficients and interpret the results. Before doing the analysis, we predict nonlinear relationships between \textit{Wage} and \textit{Age}, \textit{Education}. Finally, notations, mathematical proofs of the \acs{OLS} formula and \acs{GMT} verifications are provided in Appendix \ref{appendix:math}.
\section{Results}
\label{sec:results}

\begin{table}[htbp]\centering
\begin{adjustwidth}{-0.6in}{-0.5in}
\begin{tabular}{l*{6}{c}}
\hline\hline
Dependent variable: Wage&\multicolumn{1}{c}{(1)}&\multicolumn{1}{c}{(2)}&\multicolumn{1}{c}{(3)}&\multicolumn{1}{c}{(4)}&\multicolumn{1}{c}{(5)}&\multicolumn{1}{c}{(6)}\\
71800 observations      &\multicolumn{1}{c}{Wage}&\multicolumn{1}{c}{lnWage}&\multicolumn{1}{c}{lnWage}&\multicolumn{1}{c}{lnWage}&\multicolumn{1}{c}{lnWage}&\multicolumn{1}{c}{lnWage}\\
\specialrule{.1em}{.05em}{.05em}
Age                   &       150.6**&      0.0657**&      0.0648**&      0.0556**&      0.0650**&      0.0436**\\
                      &        (7.56)&      (0.0011)&      (0.0011)&      (0.0044)&      (0.0011)&      (0.0047)\\
[0.5em]
Age$^2$               &              &   -0.000679**&   -0.000671**&   -0.000476**&   -0.000673**&   -0.000367**\\
                      &              &   ($<$0.0001)&   ($<$0.0001)&   ($<$0.0001)&   ($<$0.0001)&      (0.0001)\\
[0.5em]
Education             &       460.1**&      0.0100**&     0.00945**&      0.0101**&      0.0096**&     0.00750**\\
                      &        (4.22)&      (0.0001)&      (0.0001)&      (0.0011)&      (0.0001)&      (0.0012)\\
[0.5em]
Male                  &     12108.0**&       0.291**&       0.291**&       0.293**&       0.341**&       0.628**\\
                      &      (177.66)&      (0.0042)&      (0.0042)&      (0.0042)&      (0.0050)&      (0.0180)\\
[0.5em]
Nonmarried            &     -9550.9**&      -0.147**&      -0.348**&      -0.146**&     -0.2151**&      -0.154**\\
                      &      (216.35)&      (0.0056)&      (0.0212)&      (0.0056)&      (0.0223)&      (0.0242)\\
[0.5em]
US                    &      1764.7**&      0.0750**&      0.0768**&      0.0756**&      0.0764**&       -0.00900\\
                      &      (229.35)&      (0.0055)&      (0.0055)&      (0.0055)&      (0.0055)&      (0.0181)\\
[0.5em]
Age$\times$Education  &              &              &              &     0.000101*&              &    0.000233**\\
                      &              &              &              &      (0.0001)&              &      (0.0001)\\
[0.5em]
Age$^2\times$Education&              &              &              & -0.00000215**&              & -0.00000333**\\
                      &              &              &              &   ($<$0.0001)&              &   ($<$0.0001)\\
[0.5em]
Education$\times$Nonmarried&         &              &     0.00226**&              &      0.0018**&     0.00117**\\
                      &              &              &      (0.0002)&              &      (0.0002)&      (0.0003)\\
[0.5em]
Education$\times$Male &              &              &              &              &              &    -0.00313**\\
                      &              &              &              &              &              &      (0.0002)\\   
[0.5em]
Education$\times$US   &              &              &              &              &              &    0.000995**\\
                      &              &              &              &              &              &      (0.0002)\\
[0.5em]
Male$\times$Nonmarried&              &              &              &              &      -0.173**&      -0.186**\\
                      &              &              &              &              &      (0.0094)&      (0.0094)\\
[0.5em]
Constant              &     -6775.9**&       8.039**&       8.109**&       8.050**&       8.071**&       8.251**\\
                      &      (533.90)&      (0.0255)&      (0.0263)&      (0.0964)&      (0.0263)&      (0.1091)\\
\specialrule{.1em}{.05em}{.05em}
\multicolumn{1}{l}{\textit{F}-Statistics} \\
\hline
(a) All variables and &        4299.5&       4367.69&       3719.21&       3409.94&       3307.16&       2329.64\\
    interactions=0    &   ($<$0.0001)&   ($<$0.0001)&   ($<$0.0001)&   ($<$0.0001)&   ($<$0.0001)&   ($<$0.0001)\\
[0.5em]
(b) Age$\times$Education,&           &              &              &         75.05&              &         52.63\\
  Age$^2\times$Education=0&          &              &              &   ($<$0.0001)&              &   ($<$0.0001)\\
[0.5em]
(c) All interactions=0&              &              &         95.26&         75.05&        214.06&        144.83\\
                      &              &              &   ($<$0.0001)&   ($<$0.0001)&   ($<$0.0001)&   ($<$0.0001)\\
[0.5em]
$R^2$                 &        0.2280&        0.2614&        0.2625&        0.2635&        0.2660&        0.2706\\
\hline\hline
\end{tabular}
\begin{tablenotes}\footnotesize
\item \textsuperscript{*}These regressions are estimated using Stata. Standard errors are given in parentheses under coefficients, and \textit{p}-values are given in parentheses under \textit{F}-statistics. Individual coefficients are statistically significant at the *5\% or **1\% significance level.
\end{tablenotes}
\end{adjustwidth}
\caption{Regression results\label{tab:regression}}
\end{table}


We prove in Appendix \ref{appendix:homoskedasticity} that our error term is heteroskedastic. Thus, we use heteroskedasticity-robust $t$ statistics to estimate the parameters after obtaining heteroskedasticity-robust standard errors for $\hat{\beta_{i}}$s. The estimated regression equations are given in Table \ref{tab:regression}, where robust standard errors are given in parentheses below coefficient estimates. We test three joint hypotheses using $F$-statistics in every regression. Our base specification, as well as the best-fitted model, is Regression (5) since the primary variables and interactions follow our economic interests and intuition. We include other regressions as our alternative specification models to see if other combinations of variables are more fitted to the data, based on plausible interactions among all variables. We can tell from $R^2$ that there is not a big difference among the regressions except for Regression (1), which we have already shown to be the least suitable model for our purposes. Therefore, we are confident that Regression (5) provides the most useful and interesting results. Though \ref{appendix:zero_cond_assump} assumption may fail in our case, the base specification may not provide a biased interpretation to a large extent. Thus, we are still positive that the regression results provide meaningful application interpretations.

In Regression (5), all variables are statistically significant at 1\% level. We reject the joint hypothesis that the coefficients of all variables and interactions are 0. The regression equation explains 26.6 percentage of the variation of \textit{lnWage}. By analyzing the regression coefficients, we find that, on average, as one's \textit{Age} increases before roughly 48 years old, \textit{Wage} will increase at a slower speed, holding other regressors equal. Following that, \textit{Wage} will decrease at a faster rate as \textit{Age} goes up. Also, we observe that the difference in \textit{Wage} between a bachelor's degree and a master's degree is 13.68\% for \textit{Nonmarried} individuals, all else equal. Similarly, there is a gender gap for \textit{Nonmarried} individuals in \textit{Wage} difference about 16.8\% on average. Interestingly, we find that the interaction coefficient for \textit{Male$\times$Nonmarried} is negative, which implies that a \textit{Nonmarried} \textit{Male}, given a 3-year-college education, has 20.71\% less \textit{Wage} compared to another \textit{Male} who is not \textit{Nonmarried}, on average. In that case, there is an economic problem: the marital status gap for \textit{Wage}. One more finding we have is that people who born in the \textit{US}, on average, tend to earn more \textit{Wage} than people who do not, given the same level of \textit{Age}, \textit{Education}, sex and marital status. Finally, we confirm that the statistics result mostly match with our predictions.

For our interest, we hypothesize two individuals for \textit{Wage} prediction using the best-fitted model. One is a 21-year-old nonmarried female born outside the US earning a bachelor's degree and another is a 20-year-old nonmarried male born outside the US. The result is 26622.4 and 30336.3. We also randomly pick a Regression from (2) to (6) to predict, and we obtain a result of 24660.8 and 31839.4. Both results lie in our range of \textit{Wage} estimation. We are optimistic about the difference due to rounding or uncontrollable errors.

